\documentclass{article}
\usepackage[utf8]{inputenc}
\usepackage{graphicx} % For including images
\usepackage{hyperref} % For hyperlinks
\usepackage{amsmath}  % For mathematical equations
\usepackage{listings} % For code listings
\usepackage{xcolor}   % For colored text in code listings

\title{Comprehensive Python Library for Wildfire Rate of Spread Modeling}
\author{}
\date{}

\begin{document}
	
	\maketitle
	\tableofcontents
	\newpage
	
	\begin{abstract}
		This work introduces a comprehensive Python library aimed at the standardization and implementation of wildfire Rate of Spread (ROS) models. The library is designed to offer researchers and practitioners a reference open-source code for exploring and applying various ROS and fuel models, enhancing the reproducibility of research and facilitating practical applications in wildfire management.
		
		The library features individual files for renowned ROS models, all referenced. Each of these files is linked to a specific publication and serves as a reference implementation. They are designed to accurately reproduce the results and figures from their respective source articles for comparison and analysis.
		
		Complementing the ROS models, the library includes files for a range of fuel models, such as those proposed by Anderson and Scott and Burgan. These implementations are structured to align with the ROS models, ensuring uniformity and comparison across different types of wildfire scenarios.
		
		The library also offers a suite of tools and methods for testing and intercomparing ROS and fuel models, as well as performing sensitivity analysis. Users can adjust variables like wind speed, slope, fuel load, humidity, and all characteristic parameters to generate and plot results.
		
		The library introduces a generic approach for training neural networks to emulate these ROS models. This method leverages advancements in machine learning to create efficient model emulators that may be fine-tuned with observational data.
		
		The library includes functionality to test and compare the computational efficiency of the different model formulations, as well as emulators.
		
		It also incorporates test datasets for the calibration and validation of models, grounding them in real-world observations.
		
		A code generator is integrated into the library, allowing for the export of these ROS and fuel models to fire behavior simulation code such as ForeFire in order to perform 2D surface simulations using the same models.
		
		Finally, the library includes this \LaTeX{} file that dynamically documents models and their implementations. This file includes detailed descriptions that must be included in the Python model files and automatically generated figures, ensuring that the documentation remains up-to-date as new models or updates are added to the library.
	\end{abstract}
	
	\section{Introduction}
	Wildfires pose significant threats to ecosystems, property, and human life worldwide. Accurate prediction of wildfire behavior, particularly the Rate of Spread (ROS), is crucial for effective firefighting strategies and resource allocation. Numerous ROS models have been developed over the years, each with its own assumptions, applicability, and limitations. However, the lack of standardized, open-source implementations of these models hinders reproducibility and comparative analysis.
	
	This paper introduces a comprehensive Python library that standardizes the implementation of various wildfire ROS models. The library serves as a central repository of reference implementations, enabling researchers and practitioners to explore, compare, and apply different models within a consistent framework. By providing tools for sensitivity analysis, model intercomparison, and neural network emulation, the library enhances the reproducibility of wildfire research and facilitates practical applications in wildfire management.
	
	\section{Installation}
	\subsection{Prerequisites}
	Before installing the library, ensure that the following prerequisites are met:
	\begin{itemize}
		\item Python 3.7 or higher is installed.
		\item The \texttt{pip} package manager is available.
	\end{itemize}
	
	\subsection{Installation Steps}
	Follow these steps to install the library:
	
	\begin{enumerate}
		\item \textbf{Clone the Repository:}
		\begin{verbatim}
			git clone https://github.com/yourusername/wildfire_ROS_models.git
			cd wildfire_ROS_models
		\end{verbatim}
		
		\item \textbf{Create a Virtual Environment (Optional but Recommended):}
		\begin{verbatim}
			python3 -m venv venv
			source venv/bin/activate  # On Windows: venv\Scripts\activate
		\end{verbatim}
		
		\item \textbf{Install Dependencies:}
		\begin{verbatim}
			pip install -r requirements.txt
		\end{verbatim}
		
		\item \textbf{Install the Package:}
		\begin{verbatim}
			pip install --editable .
		\end{verbatim}
		The \verb!--editable! flag allows you to make changes to the code without reinstalling the package.
	\end{enumerate}
	
	\section{ROS Model Implementations}
	\subsection{Model Structure}
	The library organizes each ROS model into individual Python files, each corresponding to a specific published model. This modular structure ensures that each implementation is self-contained and directly references its original publication, serving as a transparent and verifiable source. The models are designed to reproduce the results and figures from their respective source articles, allowing users to validate and compare model outputs under various conditions.
	
	\subsection{Case Studies: Balbi and Rothermel Models}
	As examples, the library includes implementations of the Balbi (2020) and Rothermel (1972) models. The Balbi model emphasizes the physical processes of heat transfer in wildfires, while the Rothermel model provides a semi-empirical approach widely used in fire behavior prediction systems.
	
	\subsubsection{Balbi Model}
	The Balbi model focuses on the radiative and convective heat transfer mechanisms in wildfire spread. The library's implementation allows users to simulate fire spread under different fuel loads, wind speeds, and moisture contents, replicating the findings from Balbi's original work.
	
	\subsubsection{Rothermel Model}
	The Rothermel model is a cornerstone in wildfire modeling, forming the basis for several fire behavior prediction systems. The library provides a detailed implementation of the Rothermel model, including its adjustments and extensions as presented in Andrews and Rothermel (2018). Users can modify environmental parameters and fuel characteristics to study the model's behavior under various scenarios.
	
	\subsection{Adding New Models}
	The library is designed to be extensible. Adding a new ROS model involves creating a new Python file that follows the established structure:
	
	\begin{itemize}
		\item Reference the original publication.
		\item Define the input parameters and their ranges.
		\item Implement the mathematical equations of the model.
		\item Include documentation and examples.
	\end{itemize}
	
	This approach encourages contributions from the community, fostering collaboration and continuous improvement of the library.
	
	\section{Fuel Model Implementations}
	Accurate ROS predictions depend heavily on the characterization of the fuels involved. The library includes implementations of various fuel models, such as those proposed by Anderson (1982) and Scott and Burgan (2005). These fuel models provide standardized descriptions of fuel properties, such as fuel load, particle size, and moisture content.
	
	\subsection{Anderson's Fuel Models}
	Anderson's 13 fuel models categorize fuels based on their characteristics and typical fire behavior. The library includes these models, allowing users to select and modify fuel parameters corresponding to different vegetation types.
	
	\subsection{Scott and Burgan's Fuel Models}
	Scott and Burgan expanded upon Anderson's work by introducing 40 standard fire behavior fuel models. The library's implementation of these models provides a more comprehensive set of fuel descriptions, covering a wider range of vegetation types and conditions.
	
	\section{Handling Units with \texttt{model\_parameters} Class}
	\subsection{Solving Units Conversion Problems}
	One of the challenges in implementing ROS and fuel models is handling the various units of measurement used in different models and publications. The library addresses this issue with the \texttt{model\_parameters} class, which dynamically handles and converts units.
	
	The class stores all parameter values in SI units internally. When a parameter is accessed or assigned with a unit suffix, the class automatically performs the necessary conversion.
	
	\subsection{Example Usage of \texttt{model\_parameters} Class}
	Below is a sample code demonstrating how to use the \texttt{model\_parameters} class:
	
	\begin{lstlisting}[language=Python, caption=Using the model\_parameters class]
		from wildfire_ROS_models.model_parameters import model_parameters
		
		# Create an instance of model_parameters
		params = model_parameters()
		
		# Assign values with units
		params.wind_miph = 10        # Wind speed in miles per hour
		params.slope_deg = 5         # Slope in degrees
		params.fl1h_tac = 2.0        # Fuel load in tons per acre
		
		# Access values with automatic unit conversion
		wind_speed_mps = params.wind_mps   # Wind speed in meters per second
		slope_rad = params.slope_rad       # Slope in radians
		fuel_load_kgm2 = params.fl1h_kgm2  # Fuel load in kilograms per square meter
		
		print(f"Wind speed: {wind_speed_mps:.2f} m/s")
		print(f"Slope: {slope_rad:.4f} radians")
		print(f"Fuel load: {fuel_load_kgm2:.2f} kg/m^2")
	\end{lstlisting}
	
	\section{Testing and Intercomparison Tools}
	\subsection{Variable Adjustments}
	The library offers tools for users to adjust key variables affecting fire spread, such as wind speed, slope angle, fuel load, and moisture content. By manipulating these parameters, users can observe how different models respond to changes in environmental conditions.
	
	\subsection{Comparative Analysis}
	To facilitate comparative studies, the library includes functions for running multiple models simultaneously and plotting their outputs on the same graph. This feature allows users to directly compare the ROS predictions of different models under identical conditions, highlighting their relative sensitivities and performance.
	
	\subsection{Sample Code for Running Models}
	Below is an example of how to run a ROS model and plot the results:
	
	\begin{lstlisting}[language=Python, caption=Running a ROS model and plotting results]
		from wildfire_ROS_models.models import RothermelAndrews2018
		from wildfire_ROS_models.runROS import run_model, plot_results
		
		# Define model parameters
		parameters = {
			'wind_miph': 0,   # Wind speed in miles per hour
			'slope_deg': 0,   # Slope in degrees
			'fuel_model': '2' # Fuel model code
		}
		
		# Run the model over a range of wind speeds
		results = run_model(
		model_name="RothermelAndrews2018",
		params=parameters,
		var_name="wind_miph",
		var_values=range(0, 20, 1)
		)
		
		# Plot the results
		plot_results([results], x_var='wind_miph', y_var='ROS_ftmin')
	\end{lstlisting}
	
	\subsection{Including Sensitivity Analysis Figure}
	The library includes tools for performing sensitivity analysis using Sobol methods. An example of a sensitivity analysis plot is shown in Figure \ref{fig:sobol}.
	
	\begin{figure}[h]
		\centering
		\includegraphics[width=0.8\textwidth]{sobol.png}
		\caption{Sobol sensitivity analysis results for the RothermelAndrews2018 model.}
		\label{fig:sobol}
	\end{figure}
	
	\section{Neural Network Emulator for ROS Models}
	Recognizing the potential of machine learning in wildfire modeling, the library introduces a generic approach for training neural networks to emulate existing ROS models. This method enables the creation of efficient model emulators that can approximate complex ROS models with reduced computational costs. These emulators can also be fine-tuned using observational data, enhancing their accuracy and applicability.
	
	\subsection{Training a Neural Network Emulator}
	The library provides scripts and functions to train a neural network emulator. Below is an example command to train an emulator for the RothermelAndrews2018 model:
	
	\begin{verbatim}
		python train_emulator.py --target_ros_model RothermelAndrews2018 --n_samples 32768
	\end{verbatim}
	
	\section{Computational Efficiency}
	The library includes functionality to test and compare the computational efficiency of different model formulations and their neural network emulators. By benchmarking execution times, users can assess the trade-offs between model complexity and computational speed, aiding in the selection of appropriate models for real-time applications.
	
	\section{Code Generation and Model Export}
	An integrated code generator allows users to export ROS and fuel models into C++ code compatible with simulation platforms like ForeFire. This feature facilitates the incorporation of the library's models into larger wildfire simulation frameworks, enabling 2D surface simulations using the same underlying models.
	
	\subsection{Exporting Models to C++}
	To export a model to C++ code for use in ForeFire, use the following command:
	
	\begin{verbatim}
		python py_ROS_models_to_forefire_cpp.py --model RothermelAndrews2018 --output_path ./exported_models/
	\end{verbatim}
	
	\section{Empirical Data Integration}
	To ground the models in real-world observations, the library incorporates test datasets for model calibration and validation. These datasets include empirical measurements of fire spread under various conditions, providing a basis for assessing model accuracy and refining model parameters.
	
	\section{Dynamic Documentation with \LaTeX}
	The library includes a dynamic \LaTeX{} documentation system that automatically updates with model descriptions and generated figures. By extracting information directly from the Python model files and including results from test scripts, the documentation remains current as new models or updates are added. This approach ensures that users have access to detailed explanations and visualizations of each model's behavior.
	
	\subsection{Structure and Content}
	The documentation is structured to mirror the organization of the library, with sections dedicated to ROS models, fuel models, and tools. Model descriptions and equations are included directly from the source code documentation, ensuring consistency and reducing redundancy.
	
	\subsection{Automatic Figure Generation}
	Figures, such as model outputs and sensitivity analysis plots, are generated by test scripts and included in the documentation. This ensures that the figures are always up-to-date with the latest code changes.
	
	\section{Conclusion and Future Work}
	The comprehensive Python library presented in this work serves as a valuable resource for the wildfire modeling community. By standardizing the implementation of various ROS and fuel models, the library enhances reproducibility and facilitates comparative analysis. The inclusion of tools for sensitivity analysis, neural network emulation, and code export broadens the library's applicability in both research and operational contexts.
	
	Future developments may include the incorporation of additional ROS and fuel models, integration with more simulation platforms, and the expansion of empirical datasets for model validation. Continued collaboration and contributions from the community will further strengthen the library's utility and impact in wildfire research and management.
	
	\begin{thebibliography}{9}
		\bibitem{anderson1982} 
		H. E. Anderson, 
		\textit{Aids to Determining Fuel Models for Estimating Fire Behavior}, 
		USDA Forest Service, Intermountain Forest and Range Experiment Station, General Technical Report INT-122, 1982.
		
		\bibitem{scott2005} 
		J. H. Scott and R. E. Burgan, 
		\textit{Standard Fire Behavior Fuel Models: A Comprehensive Set for Use with Rothermel's Surface Fire Spread Model}, 
		USDA Forest Service, Rocky Mountain Research Station, General Technical Report RMRS-GTR-153, 2005.
		
		\bibitem{rothermel1972} 
		R. C. Rothermel, 
		\textit{A Mathematical Model for Predicting Fire Spread in Wildland Fuels}, 
		USDA Forest Service, Intermountain Forest and Range Experiment Station, Research Paper INT-115, 1972.
		
		\bibitem{balbi2020} 
		J. H. Balbi et al., 
		\textit{A Physical Model for Wildfire Rate of Spread Predictions in the Field}, 
		Combustion Science and Technology, vol. 192, no. 11, pp. 1996–2015, 2020.
		
		\bibitem{andrews2018} 
		P. L. Andrews, 
		\textit{The Rothermel Surface Fire Spread Model and Associated Developments: A Comprehensive Explanation}, 
		USDA Forest Service, Rocky Mountain Research Station, General Technical Report RMRS-GTR-371, 2018.
		
		\bibitem{vacchiano2015} 
		G. Vacchiano and D. Ascoli, 
		\textit{An Implementation of the Rothermel Fire Spread Model in the R Programming Language}, 
		Fire Technology, vol. 51, pp. 523–535, 2015.
		
	\end{thebibliography}
	
\end{document}